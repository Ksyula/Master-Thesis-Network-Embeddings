\chapter*{Zusammenfassung}
\label{cha:zusammenfassung}
Die Beziehungen zwischen verschiedenen Beteiligten der Finanzindustrie werden häufig in einem komplexen finanziellen Netzwerk modelliert. Solche Netzwerke bestehen aus Teilnehmer, die miteinander mit Kanten, die die Transaktionen repräsentieren, verbunden sind. Die Teilnehmer sind beispielweise Kunden und Firmen im Kontext des Privatkunden-Bankings oder Individuen und juristische Personen im Kontext der Interbankennetze. Weil die finanziellen Daten sensibel sind, ist der Umgang damit streng reguliert, was fast immer die umfassende multivariate Analyse der Finanzdaten durch Mangel an Charakteristiken vermeidet. Trotzdem bleiben die strukturellen Merkmale dieser Daten zugreifbar, was die Exploration anonymisierter finanziellen Daten motiviert. Moderne Frameworks für schnelles Network Embedding leiten strukturelle Eigenschaften von lokalen Knotennachbarschaften ab und finden eine Funktion, die die Knoten in die niedrigdimensionalen Räume abbildet. Jedoch verarbeitet die Mehrheit der Network Embedding Methoden nur statische Netzwerke, wenn die meisten finanziellen Netzwerke dynamisch sind.

Diese Masterarbeit präsentiert ein Konzept der unbeobachteten Knotensegmentation in finanziellen Netzwerken. Das nutzt das Framework des Network Embedding für das Lernen struktureller Merkmale dynamischer Netzwerke. Dieses Konzept wurde in der Form einer Datenverarbeitung-Pipeline, die für ein Eingabe-Dataset mehrere Zwischenergebnisse erzeugt, wobei jedes Zwischenergebnis eine Menge der Teilnehmergruppen darstellt. Jede Teilnehmergruppe beinhaltet die Teilnehmer, die ähnlich im Sinne der definierten Merkmale sind. Aus diesen Ausgaben wird ein Ergebnis anhand einer Evaluation der Zwischenergebnisse ausgewählt.

Diese Arbeit besteht aus der Zusammenfassung früherer Studien, Entwicklung eines Konzeptes und prototypischen Implementierung, Beschreibung der Experimente mit echten Daten und Evaluierung des Konzepts auf einem dynamischen finanziellen Netzwerk. Die Experimente zeigen die Fähigkeit der Pipeline, neue Kenntnisse aus dem Eingabenetzwerk basierend auf den strukturellen Merkmalen zu gewinnen.
